\documentclass{beamer}
\usepackage[latin1]{inputenc}
\usepackage{amssymb,amsmath,latexsym,listings,float,algorithm}
\usepackage[noend]{algorithmic}
\lstset{
 basicstyle = \ttfamily,
 basewidth  = {.5em,0.4em},
 columns    = flexible,
 % columns  = fullflexible,
}

\usetheme{Warsaw}
\title[YAGL]{YAGL\\Yet Another Graphics Language}
\author{Edgar Aroutiounian \\ Jeff Barg \\ Robert Cohen}
\institute{COMS S4115 - Programming Languages \& Translators \\ Columbia University}
\date{August 15, 2014}

\lstset{
 basicstyle = \ttfamily,
 basewidth  = {.5em,0.4em},
 columns    = flexible,
 % columns  = fullflexible,
}

\begin{document}

\begin{frame}
\titlepage
\end{frame}


\begin{frame}{What is YAGL?}
YAGL is a new programming language for constructing graphics, with the intent of allowing programmers to construct SVGs. The language also features a static type system to help with potential type errors. \\

\end{frame}

\begin{frame}{How Does YAGL work?}
The YAGL internal stack begins with the a lexical tokenizer. The tokenizer converts a stream of chars into a 
list of tokens in accordance with YAGL's grammar. The resultant list is then passed to a parser which creates an 
abstract syntax tree. This AST is then passed to a 
semantic analyzer which stops the compilation process
upon encountering a semantic error, i.e. adding a String to an Integer. If the semantic analysis succeeds, the AST 
passes to the code generator. At this stage, code generation starts and outputs a single C++ source code file.
\end{frame}

\begin{frame}{YAGL Demonstration}
\begin{lstlisting}
func main() { 
 Int a = numberOfRings();
 Int b = a;
 String c = "red";
 Int d = 0;

 canvas(500, 500);

 while (a) {
   if ((a/2 + a/2) == a) {
     c = "red";
   } else {
     c = "blue";
   }

   d = (250 * a) / b;

   addCircle(d, 250, 250, c, c);

   a = a - 1;
 }

 text("A Target!", 40, 40, 300);
}

func numberOfRings() {
 return 3;
}
\end{lstlisting}
s
\end{frame}

\begin{frame}{What Problems Did We Run Into}
Architecture
\end{frame}

\end{document}